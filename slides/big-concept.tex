% \begin{frame}
% \frametitle{Big concept}
% \begin{columns}[T]
%     \begin{column}{0.4\textwidth}
%         \begin{tikzpicture}
%             \fill[lightblue] (0,0) rectangle (4,4);
%             \node[white] at (2,2) {\Huge \faRocket};
%         \end{tikzpicture}
%     \end{column}
%     \begin{column}{0.6\textwidth}
%         \LARGE Bring the attention of your audience over a key concept using icons or illustrations
%     \end{column}
% \end{columns}
% \end{frame}

\begin{frame}
\frametitle{What is CanSat?}
\begin{columns}[T]
    \begin{column}{0.4\textwidth}
        \begin{tikzpicture}
            \fill[lightblue] (0,0) rectangle (4,4);
            \node[white] at (2,2) {\Huge \faSatellite};
            \node[white] at (2,1) {\small A satellite in a can};
        \end{tikzpicture}
    \end{column}
    \begin{column}{0.6\textwidth}
        \LARGE A CanSat combines all major subsystems found in a satellite into the size of a soda can:\\[1em]
        \normalsize
        \begin{itemize}
            \item Power systems
            \item Communications
            \item Sensors
            \item Recovery system
        \end{itemize}
    \end{column}
\end{columns}
\end{frame}
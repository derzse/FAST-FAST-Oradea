% \begin{frame}
% \frametitle{Ce sunt competițiile spațiale?}

% \begin{columns}[T]
%     \begin{column}{0.45\textwidth}
%         \textcolor{magenta}{\textbf{Caracteristicile competițiilor spațiale}}
%         \vspace{0.3cm}
%     \end{column}
%     \begin{column}{0.45\textwidth}
%         \textcolor{magenta}{\textbf{Beneficiile participării}}
%         \vspace{0.3cm}
%     \end{column}
% \end{columns}

% \begin{columns}[T]
%     \begin{column}{0.48\textwidth}
%         {\usebeamerfont{column text}
%         \begin{itemize}
%             \item \textbf{Programe educaționale interactive} care provoacă elevii să rezolve probleme legate de spațiu
%         \end{itemize}
%         }
%     \end{column}
%     \begin{column}{0.48\textwidth}
%         {\usebeamerfont{column text}
%         \begin{itemize}
%             \item \textbf{Dezvoltă abilități practice} în inginerie, fizică și lucru în echipă
%         \end{itemize}
%         }
%     \end{column}
% \end{columns}
% \vspace{0.3cm}
% \begin{columns}[T]
%     \begin{column}{0.48\textwidth}
%         {\usebeamerfont{column text}
%         \begin{itemize}
%             \item \textbf{Activități interdisciplinare} ce combină știința, tehnologia, ingineria și matematica
%         \end{itemize}
%         }
%     \end{column}
%     \begin{column}{0.48\textwidth}
%         {\usebeamerfont{column text}
%         \begin{itemize}
%             \item \textbf{Fac legătura între teorie și aplicarea practică}, consolidând cunoștințele
%         \end{itemize}
%         }
%     \end{column}
% \end{columns}
% \vspace{0.3cm}
% \begin{columns}[T]
%     \begin{column}{0.48\textwidth}
%         {\usebeamerfont{column text}
%         \begin{itemize}
%             \item \textbf{Proiecte practice} ce implică designul și construirea de roboți sau vehicule spațiale
%         \end{itemize}
%         }
%     \end{column}
%     \begin{column}{0.48\textwidth}
%         {\usebeamerfont{column text}
%         \begin{itemize}
%             \item \textbf{Pregătesc următoarea generație de profesioniști} din industria spațială
%         \end{itemize}
%         }
%     \end{column}
% \end{columns}
% \vspace{0.3cm}
% \begin{columns}[T]
%     \begin{column}{0.48\textwidth}
%         {\usebeamerfont{column text}
%         \begin{itemize}
%             \item \textbf{Colaborare internațională} prin care elevii interacționează cu participanți din alte țări
%         \end{itemize}
%         }
%     \end{column}
%     \begin{column}{0.48\textwidth}
%         {\usebeamerfont{column text}
%         \begin{itemize}
%             \item \textbf{Îmbunătățesc gândirea critică} și capacitatea de rezolvare a problemelor complexe
%         \end{itemize}
%         }
%     \end{column}
% \end{columns}

% \bcolorbar[-0.125cm]
% \end{frame}







% \vfill
% {\usebeamerfont{column text}\textcolor{magenta}{More info on how to use this template at \underline{www.slidescarnival.com/help-use-presentation-template}}}

% \vspace{0.2cm}
% {\usebeamerfont{column text}This template is free to use under \underline{Creative Commons Attribution license.} You can keep the Credits slide or mention SlidesCarnival and other resources used in a slide footer.}

% \bcolorbar[-0.125cm]
% \end{frame}




% \begin{frame}
% \frametitle{Provocări Tehnice}

% \textbf{Misiunea Primară: Cerințe esențiale}

% \begin{itemize}
%     \item \textbf{Transmiterea telemetriei}: Asigurarea unei legături de comunicație fiabile pentru a transmite datele colectate în timp real către stația de sol. Aceasta implică proiectarea și implementarea unui sistem de radio comunicații eficient și robust.
    
%     \item \textbf{Coborâre controlată}: Implementarea unui mecanism care să permită CanSat-ului să coboare într-un mod stabil și previzibil. Soluțiile pot include parașute, rotoare sau alte dispozitive de frânare care să mențină viteza de coborâre în limite sigure.
    
%     \item \textbf{Colectarea datelor}: Monitorizarea și înregistrarea parametrilor atmosferici precum temperatura și presiunea pe parcursul coborârii. Aceste date sunt esențiale pentru înțelegerea mediului și pentru validarea funcționării senzorilor.
% \end{itemize}

% \end{frame}

% \begin{frame}
% \frametitle{Provocări Tehnice (continuare)}

% \textbf{Misiunea Secundară: Obiective specifice echipei}

% \begin{itemize}
%     \item \textbf{Senzorizare avansată}: Integrarea de senzori suplimentari pentru a colecta date complexe, cum ar fi umiditatea, nivelurile de CO$_2$, radiațiile UV sau câmpurile magnetice. Acest lucru permite echipelor să exploreze aspecte științifice mai aprofundate.
    
%     \item \textbf{Cerințe specifice de aterizare}: Proiectarea unui sistem care să permită aterizarea CanSat-ului într-o poziție sau locație specifică. De exemplu, atingerea unei zone țintă sau aterizarea în poziție verticală pentru a facilita recuperarea.
    
%     \item \textbf{Experimente suplimentare}: Dezvoltarea și implementarea de experimente inovatoare, cum ar fi:
%     \begin{itemize}
%         \item \textbf{Eliberarea unui rover miniatural} care poate efectua explorări la sol.
%         \item \textbf{Capturarea de imagini sau video} în timpul coborârii pentru analiză ulterioară.
%         \item \textbf{Testarea unor materiale sau componente noi} în condiții reale de zbor.
%     \end{itemize}
% \end{itemize}

% \end{frame}







% \begin{frame}
% \frametitle{Rezultate ale Învățării}

% \begin{itemize}
%     \item<1-> \textbf{Inginerie a Sistemelor}: Dezvoltarea unei înțelegeri profunde a modului în care diferite componente ale unui sistem tehnic interacționează și se integrează. Elevii învață să proiecteze, să analizeze și să optimizeze sisteme complexe, abordând probleme multidisciplinare și găsind soluții eficiente.

%     \item<-2> \textbf{Management de Proiect}: Dobândirea abilităților esențiale pentru planificarea și coordonarea unui proiect de la concept până la finalizare. Aceasta include gestionarea resurselor, stabilirea obiectivelor, monitorizarea progresului și adaptarea la schimbări pentru a asigura succesul proiectului.

%     \item<-3> \textbf{Documentație Tehnică}: Învățarea modului de a crea documentație tehnică profesională, precum rapoarte, planuri și specificații. Aceasta este crucială pentru comunicarea eficientă între membrii echipei și cu părțile interesate, precum și pentru conformitatea cu standardele și reglementările din industrie.
% \end{itemize}

% \end{frame}

% \begin{frame}
% \frametitle{Rezultate ale Învățării}

% \begin{itemize}
%     \item \textbf{Abilități Practice de Inginerie}: Aplicarea cunoștințelor teoretice în situații practice prin construirea, testarea și calibrarea echipamentelor și sistemelor. Elevii își dezvoltă competențele tehnice prin experiență directă, învățând să rezolve probleme reale și să opereze cu instrumente și tehnologii specifice.

%     \item \textbf{Colaborare în Echipa}: Îmbunătățirea abilităților de a lucra eficient într-o echipă, promovând comunicarea deschisă, respectul reciproc și împărțirea responsabilităților. Elevii învață să valorifice punctele forte ale fiecărui membru pentru a atinge obiective comune.

%     \item \textbf{Abilități de Prezentare}: Dezvoltarea capacității de a comunica idei și rezultate în mod clar și convingător, atât verbal cât și în scris. Elevii își perfecționează abilitățile de prezentare prin expuneri în fața publicului și feedback constructiv, pregătindu-se pentru interacțiuni profesionale viitoare.
% \end{itemize}

% \end{frame}




% \begin{frame}
% \frametitle{Rezultate ale Învățării - Partea 1}

% \begin{itemize}[<+->] % This makes items appear one at a time
%     \item \textbf{Inginerie a Sistemelor}
%         \begin{itemize}
%             \item Dezvoltarea înțelegerii sistemelor tehnice integrate
%             \item Proiectare și analiză de sisteme complexe
%             \item Abordarea problemelor multidisciplinare
%         \end{itemize}
    
%     \item \textbf{Management de Proiect}
%         \begin{itemize}
%             \item Planificare și coordonare de la concept la finalizare
%             \item Gestionarea resurselor și stabilirea obiectivelor
%             \item Monitorizarea progresului și adaptarea la schimbări
%         \end{itemize}
    
%     \item \textbf{Documentație Tehnică}
%         \begin{itemize}
%             \item Crearea documentației tehnice profesionale
%             \item Comunicare eficientă între membrii echipei
%             \item Conformitate cu standardele din industrie
%         \end{itemize}
% \end{itemize}

% \bcolorbar
% \end{frame}

% \begin{frame}
% \frametitle{Rezultate ale Învățării - Partea 2}

% \begin{itemize}[<+->] % This makes items appear one at a time
%     \item \textbf{Abilități Practice de Inginerie}
%         \begin{itemize}
%             \item Aplicarea cunoștințelor teoretice în practică
%             \item Construirea și testarea sistemelor
%             \item Dezvoltarea competențelor tehnice prin experiență directă
%         \end{itemize}
    
%     \item \textbf{Colaborare în Echipă}
%         \begin{itemize}
%             \item Comunicare deschisă și respect reciproc
%             \item Împărțirea responsabilităților
%             \item Valorificarea punctelor forte ale membrilor
%         \end{itemize}
    
%     \item \textbf{Abilități de Prezentare}
%         \begin{itemize}
%             \item Comunicare clară a ideilor și rezultatelor
%             \item Perfecționarea prezentărilor publice
%             \item Pregătirea pentru interacțiuni profesionale
%         \end{itemize}
% \end{itemize}

% \bcolorbar
% \end{frame}

% % Alternative version with visual elements
% \begin{frame}
% \frametitle{Rezultate ale Învățării - Sinteză}

% \begin{columns}[T]
%     \begin{column}{0.48\textwidth}
%         \begin{itemize}[<+->]
%             \item \textbf{\textcolor{mainblue}{Competențe Tehnice}}
%                 \begin{itemize}
%                     \item Inginerie a Sistemelor
%                     \item Abilități Practice
%                     \item Documentație Tehnică
%                 \end{itemize}
%         \end{itemize}
%     \end{column}
%     \begin{column}{0.48\textwidth}
%         \begin{itemize}[<+->]
%             \item \textbf{\textcolor{magenta}{Competențe Soft}}
%                 \begin{itemize}
%                     \item Management de Proiect
%                     \item Colaborare în Echipă
%                     \item Abilități de Prezentare
%                 \end{itemize}
%         \end{itemize}
%     \end{column}
% \end{columns}

% \onslide<3->{
% \begin{center}
% \begin{tikzpicture}[
%     box/.style={minimum width=\textwidth-2cm,minimum height=1.2cm}
% ]
%     \node[box,fill=yellow] {
%         \begin{minipage}{\textwidth-3cm}
%             \centering
%             \small
%             Dezvoltarea completă a viitorului inginer spațial
%         \end{minipage}
%     };
% \end{tikzpicture}
% \end{center}
% }

% \bcolorbar
% \end{frame}

% \begin{frame}
% \frametitle{Rezultate ale Învățării - Sinteză}
% \begin{columns}[T]
%     \begin{column}{0.45\textwidth}
%         \begin{itemize}
%             \item<1-> \textbf{\textcolor{mainblue}{Inginerie a Sistemelor}}
%             \item<2-> \textbf{\textcolor{mainblue}{Abilități Practice}}
%             \item<3-> \textbf{\textcolor{mainblue}{Documentație Tehnică}}
%         \end{itemize}
%     \end{column}
%     \begin{column}{0.5\textwidth}
%         \visible<1>{
%             \textcolor{magenta}{\textbf{Dezvoltarea Sistemelor}}\\[0.5em]
%             {\small Înțelegerea profundă a modului în care diferite componente interacționează și se integrează. Proiectare, analiză și optimizare de sisteme complexe pentru misiuni spațiale.}
%         }
%         \visible<2>{
%             \textcolor{magenta}{\textbf{Aplicare Practică}}\\[0.5em]
%             {\small Experiență directă în construirea și testarea echipamentelor. Dezvoltarea competențelor tehnice prin rezolvarea problemelor reale și operarea cu tehnologii specifice.}
%         }
%         \visible<3>{
%             \textcolor{magenta}{\textbf{Comunicare Tehnică}}\\[0.5em]
%             {\small Crearea documentației profesionale precum rapoarte, planuri și specificații. Esențial pentru comunicarea eficientă și conformitatea cu standardele din industrie.}
%         }
%     \end{column}
% \end{columns}

% \bcolorbar
% \end{frame}















% \begin{frame}
% \frametitle{Provocări Tehnice}
% \textcolor{yellow}{\Large{\textbf{Cerințe și Oportunități}}}
% \begin{columns}[T]
%     \begin{column}{0.45\textwidth}
%         \begin{itemize}
%             \item<1-> \textbf{\textcolor{mainblue}{Misiune Primară}}
%                 \begin{itemize}
%                     \item<1-> Telemetrie
%                     \item<2-> Coborâre
%                     \item<3-> Date de bază
%                 \end{itemize}
%             \vspace{0.5cm}
%             \item<4-> \textbf{\textcolor{mainblue}{Misiune Secundară}}
%                 \begin{itemize}
%                     \item<4-> Senzori avansați
%                     \item<5-> Aterizare precisă
%                     \item<6-> Experimente speciale
%                 \end{itemize}
%         \end{itemize}
%     \end{column}
%     \begin{column}{0.5\textwidth}
%         \only<1>{\parbox[t]{\textwidth}{
%             \textcolor{magenta}{\textbf{Sistem de Comunicații}}\\[0.5em]
%             {\small Implementarea unei legături radio fiabile pentru transmiterea datelor în timp real către stația de sol. Sistem robust de comunicații pentru monitorizare continuă.}
%         }}
%         \only<2>{\parbox[t]{\textwidth}{
%             \textcolor{magenta}{\textbf{Control al Coborârii}}\\[0.5em]
%             {\small Mecanism de coborâre stabilă și controlată folosind parașute sau alte sisteme de frânare. Menținerea vitezei în limite sigure pentru misiune.}
%         }}
%         \only<3>{\parbox[t]{\textwidth}{
%             \textcolor{magenta}{\textbf{Senzori de Bază}}\\[0.5em]
%             {\small Monitorizarea parametrilor atmosferici esențiali: temperatură, presiune, altitudine. Date cruciale pentru validarea funcționării sistemelor.}
%         }}
%         \only<4>{\parbox[t]{\textwidth}{
%             \textcolor{magenta}{\textbf{Măsurători Complexe}}\\[0.5em]
%             {\small Integrarea senzorilor avansați pentru umiditate, CO$_2$, UV, câmpuri magnetice. Explorarea aspectelor științifice aprofundate.}
%         }}
%         \only<5>{\parbox[t]{\textwidth}{
%             \textcolor{magenta}{\textbf{Landing Precis}}\\[0.5em]
%             {\small Sisteme pentru aterizare în poziție și locație specifică. Optimizarea recuperării și colectării datelor post-misiune.}
%         }}
%         \only<6>{\parbox[t]{\textwidth}{
%             \textcolor{magenta}{\textbf{Inovații}}\\[0.5em]
%             {\small • Rover miniatural pentru explorare\\
%             • Captură foto/video în timpul zborului\\
%             • Testarea de materiale și componente noi}
%         }}
%     \end{column}
% \end{columns}

% \bcolorbar
% \end{frame}

